\documentclass{article}

\usepackage{url}
\usepackage{hyperref}
\usepackage{xstring}
\usepackage{etoolbox}

% ---- URL citation macros ----


\newcounter{citeurlcounter}
% List to store cited URLs
\newcommand{\citedurllist}{} 

% Command to create a hypertarget at the citation point
\newcommand*\raisetarget[2][1em]{%
  \raisebox{#1}[0pt][0pt]{\hypertarget{#2}{}}%
}

% Command to cite a URL
% 1. The Wrapper: Changes # catcode so LaTeX doesn't think it's a parameter
\newcommand{\citeurl}{%
    % --- START GROUP ---
    \begingroup
    \catcode`\#=12 
    \citeurlaux
}

% 2. The Worker (HASH-BASED Deduplication)
\newcommand{\citeurlaux}[1]{%  
    % --- PRELIMINARY SETUP ---
    % Ensure URL has a protocol
    \edef\thelink{\detokenize{#1}}%
    \IfSubStr{#1}{://}{%
        \def\thelink{#1}%
    }{%
        \def\thelink{http://#1}% we assume http instead of https
    }%
    % remove trailing slash (usually they point to the same resource)
    \IfEndWith{\thelink}{/}{%
        \StrGobbleRight{\thelink}{1}[\thelink]%
    }{}%
    % --- HASH-BASED DEDUPLICATION LOGIC ---
    % Check if a command named 'urlhash@<the_url>' exists
    \ifcsdef{urlhash@\thelink}{%
        % IF EXISTS: Retrieve the ID stored in that command
        \edef\thisid{\csuse{urlhash@\thelink}}%
    }{%
        % IF NEW: Increment counter, add to list, and SAVE ID in a hash
        \refstepcounter{citeurlcounter}%
        \expandafter\listgadd\expandafter{\expandafter\citedurllist\expandafter}\expandafter{\thelink}%
        \edef\thisid{\theciteurlcounter}%
        \csxdef{urlhash@\thelink}{\thisid}% Store ID for next time
        % Initialize occurrence counter for this URL
        \csxdef{occcount@\thisid}{0}%
    }%
    % --- CITATION OCCURRENCE TRACKING ---
    % Increment occurrence count
    \ifcsdef{occcount@\thisid}{%
        \edef\currentocc{\csuse{occcount@\thisid}}%
        \edef\newocc{\number\numexpr\currentocc+1\relax}%
        \csxdef{occcount@\thisid}{\newocc}%
    }{%
        \csxdef{occcount@\thisid}{1}%
    }%
    \edef\occnum{\csuse{occcount@\thisid}}%
    % --- FIXED LOGIC STARTS HERE ---
    % Extract domain for display
    \StrBehind{\thelink}{://}[\domainname]%
    % Remove path, port, and www.
    \IfSubStr{\domainname}{/}{\StrBefore{\domainname}{/}[\domainname]}{}%
    \IfSubStr{\domainname}{:}{\StrBefore{\domainname}{:}[\domainname]}{}%
    \IfBeginWith{\domainname}{www.}{\StrBehind{\domainname}{www.}[\domainname]}{}%
    % Truncation
    \StrLen{\domainname}[\domlength]%
    \ifnum\domlength>30 
        \StrRight{\domainname}{27}[\domainname]%
        \edef\domainname{...\domainname}%
    \fi
    % --- OUTPUT THE CITATION ---
    [\raisetarget{citeurl:\thisid:\occnum}{}% backlink target
    \hyperref[it:citeurl:\thisid]{\thisid}, % citation number
    \href{\thelink}{\domainname}]% hyperlink to URL
    % --- END GROUP ---
    \endgroup
}


% OLD: Helper to convert number to letter (1->a, 2->b, etc.)
% \newcommand{\numalph}[1]{%
%     \ifcase#1\relax \or a\or b\or c\or d\or e\or f\or g\or h\or i\or j\or k\or l\or m\or n\or o\or p\or q\or r\or s\or t\or u\or v\or w\or x\or y\or z\else ?\fi%
% }
% 
\ExplSyntaxOn
% Helper to convert number to letter (1->a, 2->b, etc.)
% a,b,c,...,z,aa,ab,ac,...,az,ba,bb,...
\NewDocumentCommand{\numalph}{m}
  { \int_to_alph:n { #1 } }
\ExplSyntaxOff


% Command to show the list of cited URLs
\newcommand{\showcitedurls}{%
    \section*{Cited URLs}\label{sec:citedurls}%

    \ifdefvoid{\citedurllist}{%
            No URLs were cited.%
    }{%
        The enumeration correspond to the citations in the text.\footnote{%
                This is a nonstandard citation style chosen for convenience. 
                It allows quick referencing without the overhead of managing a \href{https://en.wikipedia.org/wiki/BibTeX}{\texttt{.bib} file}, which can be cumbersome and time-consuming in this context.}%
        \begin{enumerate}
            \newcounter{urlitemcount}%
            \renewcommand*{\do}[1]{%
                \stepcounter{urlitemcount}%
                % Check occurrence count to decide on label formatting
                \ifcsdef{occcount@\theurlitemcount}{%
                    \edef\occcount{\csuse{occcount@\theurlitemcount}}%
                    \ifnum\occcount>1
                        % Multiple occurrences: plain label, no hyperlink
                        \renewcommand*\makelabel[1]{\hfill ####1}%
                    \else
                        % Single occurrence: hyperlinked label
                        \renewcommand*\makelabel[1]{%
                            \hyperlink{citeurl:\theurlitemcount:1}{\hfill ####1}%
                        }%
                    \fi
                }{%
                    % No occurrence data: default to hyperlink
                    \renewcommand*\makelabel[1]{%
                        \hyperlink{citeurl:\theurlitemcount:1}{\hfill ####1}%
                    }%
                }%
                \item % we use the makelabel command to create a hyperlink
                \label{it:citeurl:\theenumi}%
                \url{##1}%
                % Add back-reference links for multiple occurrences
                \ifcsdef{occcount@\theenumi}{%
                    \edef\occcount{\csuse{occcount@\theenumi}}%
                    \ifnum\occcount>1
                        \textsuperscript{%
                            \count255=1
                            \loop
                                \hyperlink{citeurl:\theenumi:\the\count255}{\numalph{\count255}}%
                                \ifnum\count255<\occcount
                                    ,\hspace{1pt}%
                                \fi
                                \advance\count255 by 1
                            \ifnum\count255<\numexpr\occcount+1\relax
                            \repeat
                        }%
                    \fi
                }{}%
            }%
            \dolistloop{\citedurllist}
        \end{enumerate}
    }
}%


% ---- Document ----

\begin{document}

This document cites a website \citeurl{https://example.com}.
Another reference points to \citeurl{https://www.wikipedia.org/wiki/LaTeX}.

You can also cite the same URL again \citeurl{https://example.com} to see deduplication in action.


% % TEST CASES:
% \citeurl{www.example.com}\\
% \citeurl{https://example.com}\\
% \citeurl{https://example.com/}\\
% \citeurl{https://www.example.com}\\
% \citeurl{https://www.example.com/}\\
% \citeurl{https://www.example.com:443}\\
% \citeurl{https://www.example.com:443/a/a/a/a/a}\\
% \citeurl{https://www.example.com/a_a/}\\
% \citeurl{https://www.example.com/a/a/a/a/a}\\
% \citeurl{https://www.example.com/a/a/a/a/a/}\\
% \citeurl{https://www.example.com/aaaa/aaaa/aaaa/aaaa/aaaa/}\\
% \citeurl{https://www.example.com/a/a/a/a/a/a.html}\\
% \citeurl{https://www.example.com/a/a/a/a/a/a.pdf}\\
% \citeurl{https://www.example.com/a#top}\\
% \citeurl{https://www.example.com/a#aaaaaaaaaaaaaaaaaaaaa}\\
% \citeurl{https://www.example.com/a?query=aaa#top}\\
% \citeurl{https://www.example.com/a?query=complex+test&sort=desc&filter[]=a&filter[]=b&filter[]=c&page=42&limit=100#top}\\

% misc\\
% \citeurl{https://luminous-landscape.com/the-secret-of-the-nikon-z6-iii-camera-the-partially-stacked-sensor/}\\
% \citeurl{https://subdomain.very-long-and-unnecessarily-descriptive-registry-service.experimental-tech-branch.example.org:8080/directory/v1/assets/documentation/archived/testing/long-filename-to-check-line-breaking-capabilities.html?user=test&session=12345#deep-anchor-link}\\

% different protocols and formats\\
% \citeurl{ftp://example.com:333/}\\
% \citeurl{https://one.one.one.one}\\
% \citeurl{one.one.one.one}\\
% \citeurl{1.1.1.1}\\
% \citeurl{127.0.0.1/aaa}\\
% \citeurl{127.0.0.1:333/aaa}\\

% now duplicates:\\
% \citeurl{www.example.com}\\
% \citeurl{https://example.com}\\
% \citeurl{https://example.com/}\\
% \citeurl{https://www.example.com/a/a/a/a/a}\\
% \citeurl{https://example.com}\\
% \citeurl{https://example.com}\\
% \citeurl{https://example.com}\\
% \citeurl{https://example.com}\\
% \citeurl{https://example.com}\\
% \citeurl{https://example.com}\\
% \citeurl{https://example.com}\\
% \citeurl{https://example.com}\\
% \citeurl{https://example.com}\\
% \citeurl{https://example.com}\\
% \citeurl{https://example.com}\\
% \citeurl{https://example.com}\\
% \citeurl{https://example.com}\\
% \citeurl{https://example.com}\\
% \citeurl{https://example.com}\\
% \citeurl{https://example.com}\\
% \citeurl{https://example.com}\\
% \citeurl{https://example.com}\\
% \citeurl{https://example.com}\\
% \citeurl{https://example.com}\\
% \citeurl{https://example.com}\\
% \citeurl{https://example.com}\\
% \citeurl{https://example.com}\\

\showcitedurls

\end{document}